 
 \documentclass{article}
\usepackage{amsmath,amscd,amsbsy,amssymb,latexsym,url,bm,amsthm}
\usepackage{epsfig,graphicx,subfigure}
\usepackage{enumitem,balance}
\usepackage{wrapfig}
\usepackage{mathrsfs, euscript}
\usepackage[usenames]{xcolor}
\usepackage{hyperref}
\usepackage[vlined,ruled,commentsnumbered,linesnumbered]{algorithm2e}

\newtheorem{theorem}{Theorem}[section]
\newtheorem{lemma}[theorem]{Lemma}
\newtheorem{proposition}[theorem]{Proposition}
\newtheorem{corollary}[theorem]{Corollary}
\newtheorem{exercise}{Exercise}[section]
\newtheorem*{solution}{Solution}
\theoremstyle{definition}
\hypersetup{hidelinks}
\definecolor{steelblue}{rgb}{0.27,0.51,0.71} 

\numberwithin{equation}{section}
\numberwithin{figure}{section}

\renewcommand{\thefootnote}{\fnsymbol{footnote}}

\newcommand{\postscript}[2]
 {\setlength{\epsfxsize}{#2\hsize}
  \centerline{\epsfbox{#1}}}

\renewcommand{\baselinestretch}{1.0}


\makeatletter \renewenvironment{proof}[1][Proof] {\par\pushQED{\qed}\normalfont\topsep6\p@\@plus6\p@\relax\trivlist\item[\hskip\labelsep\bfseries#1\@addpunct{.}]\ignorespaces}{\popQED\endtrivlist\@endpefalse} \makeatother
\makeatletter
\renewenvironment{solution}[1][Solution] {\par\pushQED{\qed}\normalfont\topsep6\p@\@plus6\p@\relax\trivlist\item[\hskip\labelsep\bfseries#1\@addpunct{.}]\ignorespaces}{\popQED\endtrivlist\@endpefalse} \makeatother



\begin{document}
\noindent

%========================================================================
\noindent\framebox[\linewidth]{\shortstack[c]{
\Large{\textbf{Final Project}}\vspace{1mm}\\
CS420-Machine learning, Shikui Tu, Spring 2018}}
\begin{center}

\footnotesize{\color{steelblue}$*$ Name:Zhiwen Qiang  \quad Student ID:515030910367 \quad Email: qlightman@163.com}
\end{center}

\section{Project description.}

\section{Algorithm 1}
\subsection{Solution ideas and algorithms.}







\section{Factor Analysis (FA)}

\subsection{Pseudo-codes}
\begin{algorithm}[H]
		\BlankLine
		\SetKwInOut{Input}{input}
		\SetKwInOut{Output}{output}
		\caption{FCM algorithm}\label{FCM}
		\Input{A batch of data points.}
		\Output{Dataset being divided into several groups.}
		\BlankLine
        Initialize $U=[u_{ij}]$ matirx, $U^{0}$ \\
        
        
		\While{$||U^{k+1}-U^{k}||>=\varepsilon$}{
		     At k-step: calculate the centers vectors $C^{k}=[c_j]$ with $U^{k}$\\
		     Update $U^{k}$,$U^{k+1}$
		}
	\end{algorithm}



\section{Independent Component Analysis (ICA)}



\section{Causal discovery algorithms}
\section{Causal tree reconstruction}
\hspace{2mm}


%========================================================================
\end{document}
